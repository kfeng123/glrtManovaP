\documentclass[review]{elsarticle}

\usepackage{lineno,hyperref}
\modulolinenumbers[5]

\journal{Journal of \LaTeX\ Templates}

%%%%%%%%%%%%%%%%%%%%%%%
%% Elsevier bibliography styles
%%%%%%%%%%%%%%%%%%%%%%%
%% To change the style, put a % in front of the second line of the current style and
%% remove the % from the second line of the style you would like to use.
%%%%%%%%%%%%%%%%%%%%%%%

%% Numbered
%\bibliographystyle{model1-num-names}

%% Numbered without titles
%\bibliographystyle{model1a-num-names}

%% Harvard
%\bibliographystyle{model2-names.bst}\biboptions{authoryear}

%% Vancouver numbered
%\usepackage{numcompress}\bibliographystyle{model3-num-names}

%% Vancouver name/year
%\usepackage{numcompress}\bibliographystyle{model4-names}\biboptions{authoryear}

%% APA style
%\bibliographystyle{model5-names}\biboptions{authoryear}

%% AMA style
%\usepackage{numcompress}\bibliographystyle{model6-num-names}

%% `Elsevier LaTeX' style
\bibliographystyle{elsarticle-num}
%%%%%%%%%%%%%%%%%%%%%%%
\usepackage{xeCJK}
\usepackage{bm}
\usepackage{amsmath}
\usepackage{amssymb}
\usepackage{amsthm}
\usepackage{graphicx}
\usepackage{color}
\usepackage{booktabs}
\usepackage{algorithm}
\usepackage{algorithmic}


\DeclareMathOperator{\mytr}{tr}
\DeclareMathOperator{\mydiag}{diag}
\DeclareMathOperator{\myrank}{Rank}
\DeclareMathOperator{\myE}{E}
\DeclareMathOperator{\myVar}{Var}


\theoremstyle{plain}
\newtheorem{theorem}{\quad\quad Theorem}
\newtheorem{proposition}{\quad\quad Proposition}
\newtheorem{corollary}{\quad\quad Corollary}
\newtheorem{lemma}{Lemma}
\newtheorem{example}{Example}
\newtheorem{assumption}{\quad\quad Assumption}
\newtheorem{condition}{Condition}

\theoremstyle{definition}
\newtheorem{remark}{\quad\quad Remark}
\theoremstyle{remark}


\begin{document}

\begin{frontmatter}

\title{Elsevier \LaTeX\ template\tnoteref{mytitlenote}}
\tnotetext[mytitlenote]{Fully documented templates are available in the elsarticle package on \href{http://www.ctan.org/tex-archive/macros/latex/contrib/elsarticle}{CTAN}.}

%% Group authors per affiliation:
\author{Elsevier\fnref{myfootnote}}
\address{Radarweg 29, Amsterdam}
\fntext[myfootnote]{Since 1880.}

%% or include affiliations in footnotes:
\author[mymainaddress,mysecondaryaddress]{Elsevier Inc}
\ead[url]{www.elsevier.com}

\author[mysecondaryaddress]{Global Customer Service\corref{mycorrespondingauthor}}
\cortext[mycorrespondingauthor]{Corresponding author}
\ead{support@elsevier.com}

\address[mymainaddress]{1600 John F Kennedy Boulevard, Philadelphia}
\address[mysecondaryaddress]{360 Park Avenue South, New York}

\begin{abstract}
This template helps you to create a properly formatted \LaTeX\ manuscript.
\end{abstract}

\begin{keyword}
\texttt{elsarticle.cls}\sep \LaTeX\sep Elsevier \sep template
\MSC[2010] 00-01\sep  99-00
\end{keyword}

\end{frontmatter}

\linenumbers
\section{GLRT}
Suppose $\{X_{i1},\ldots, X_{in_i}\}$ are i.i.d.\ distributed as $N(\mu_i,\Sigma)$ for $1\leq i\leq K$.
Let $\mathbf{X}_i=(X_{i1},\ldots,X_{in_i})$ for $i=1,\ldots,k$.
The $k$ samples are independent.
$\mu_i$, $i=1\ldots, k$ and $\Sigma>0$ are unknown. An interesting problem in multivariate analysis is to test the hypotheses
\begin{equation}
    H: \mu_1=\mu_2=\cdots=\mu_k\quad v.s.\quad K: \textrm{$\mu_i\neq \mu_j$ for some $i\neq j$}.
\end{equation}
Let $\mathbf{Z}=(X_1,\ldots,X_k)$.
$$
f(Z;\mu_1,\ldots,\mu_k,\Sigma)=\prod_{i=1}^k\Big[
    (2\pi)^{-n_i p/2}|\Sigma|^{-n_i/2}\exp(-\frac{1}{2}\mathrm{tr}\Sigma^{-1}\sum_{j=1}^{n_i}(x_{ij}-\mu_i)(x_{ij}-\mu_i)^T)
    \Big].
$$
Assume $n=\sum_{i=1}^p n_i<p$. Let $a\in \mathbb{R}^p$ be a vector satisfying $a^T a=1$. Then
$$
f_a(a^T Z;\mu_1,\ldots,\mu_k,\Sigma)=
    (2\pi)^{-n/2}|a^T \Sigma a|^{-n/2}\exp\Big(-\frac{1}{2 a^T \Sigma a}\sum_{i=1}^k\sum_{j=1}^{n_i}(a^Tx_{ij}-a^T\mu_i)^2\Big)
$$
\begin{equation}
    \begin{aligned}
        \max_{\mu_1,\ldots,\mu_k,\Sigma}f_a(a^T Z,\mu_1,\ldots,\mu_k,\Sigma)
        =
        (2\pi)^{-n/2}\big(\sum_{i=1}^k\sum_{j=1}^{n_i}(a^Tx_{ij}-a^T\bar{\mathbf{X}}_i)^2\big)^{-n/2}e^{-{n}/{2}}
    \end{aligned}
\end{equation}

Let $S_i=\sum_{j=1}^{n_i}(x_{ij}-\bar{\mathbf{X}}_i)(x_{ij}-\bar{\mathbf{X}}_i)^T$ and $S=\sum_{i=1}^k S_i$.


Under $H$, we have
\begin{equation}
    \begin{aligned}
        \max_{\mu,\Sigma}f_a(a^T Z,\mu,\ldots,\mu,\Sigma)
        =
        (2\pi)^{-n/2}\big(\sum_{i=1}^k\sum_{j=1}^{n_i}(a^Tx_{ij}-a^T\bar{\mathbf{X}})^2\big)^{-n/2}e^{-{n}/{2}}
    \end{aligned}
\end{equation}

The generalized likelihood ratio test statistic is defined as
\begin{equation}
    \begin{aligned}
        T(Z)
        =
        \max_{a^T a=1, a^T S a=0} 
        a^T \sum_{i=1}^k n_i (\bar{\mathbf{X}}_i-\bar{\mathbf{X}})(\bar{\mathbf{X}}_i-\bar{\mathbf{X}})^T a
    \end{aligned}
\end{equation}
Let $J=\mathrm{diag}(n_1^{-1/2}\mathbf{1}_{n_1},\ldots,n_k^{-1/2}\mathbf{1}_{n_k})$.
Then $S=Z(I_n-JJ^T)Z^T$ and
\begin{equation}
    \begin{aligned}
        \sum_{i=1}^k n_i (\bar{\mathbf{X}}_i-\bar{\mathbf{X}})(\bar{\mathbf{X}}_i-\bar{\mathbf{X}})^T 
        =Z(JJ^T-\frac{1}{n}\mathbf{1}_n\mathbf{1}_n^T)Z^T.
    \end{aligned}
\end{equation}
The matrix $I_n-JJ^T$, $JJ^T-\frac{1}{n}\mathbf{1}_n\mathbf{1}_n^T$ and $\frac{1}{n}\mathbf{1}_n\mathbf{1}_n^T$ are all projection matrix and pairwise orthogonal with rank $n-k$, $k-1$ and $1$.

Let $\tilde{J}$ be a $n\times (n-k)$ matrix satisfied $\tilde{J}\tilde{J}^T =I-JJ^T$.
Then $S=Z\tilde{J}\tilde{J}^T Z^T$ and
 Note that 
$$
Z(JJ^T-\frac{1}{n}\mathbf{1}_n\mathbf{1}_n^T)Z^T
=ZJ(I_k-\frac{1}{n}J^T\mathbf{1}_n \mathbf{1}_n^T J)J^T Z^T.
$$
Note that $I_k-\frac{1}{n}J^T\mathbf{1}_n \mathbf{1}_n^T J$ is a projection matrix with rank $k-1$.
Let $C$ be a $k\times (k-1)$ matrix satisfied $CC^T=I_k-\frac{1}{n}J^T\mathbf{1}_n \mathbf{1}_n^T J$.

In Proposition~\ref{optProp}, letting $A=Z\tilde{J}$ and $B=ZJ C C^T J^TZ^T$ yields 
$$
\begin{aligned}
    T(Z)&=\lambda_{max}\big((I_p-
    Z\tilde{J}{(\tilde{J}^T Z^T Z\tilde{J})}^{-1}\tilde{J}^T Z^T
    )ZJCC^TJ^TZ^T (I_p-
    Z\tilde{J}{(\tilde{J}^T Z^T Z\tilde{J})}^{-1}\tilde{J}^T Z^T
    )\big)
\\
    &=\lambda_{max}\big(C^TJ^TZ^T (I_p-
    Z\tilde{J}{(\tilde{J}^T Z^T Z\tilde{J})}^{-1}\tilde{J}^T Z^T
    )ZJC\big).
\end{aligned}
$$
%where $H_A=Z\tilde{J}{(\tilde{J}^T Z^T Z\tilde{J})}^{-1}\tilde{J}^T Z^T$.
Note that
\begin{equation}
    \begin{aligned}
        &\Big(
        \begin{pmatrix}
            J^T\\
            \tilde{J}^T
        \end{pmatrix}
        Z^T Z
        \begin{pmatrix}
            J&\tilde{J}
        \end{pmatrix}
        \Big)^{-1}\\
        &=
        \begin{pmatrix}
            J^T Z^T ZJ & J^T Z^T Z\tilde{J}\\
            \tilde{J}^T Z^T ZJ & \tilde{J}^T Z^T Z \tilde{J}
        \end{pmatrix}^{-1}
        &=
        \begin{pmatrix}
            J^T {(Z^T Z)}^{-1}J & J^T {(Z^T Z)}^{-1}\tilde{J}\\
            \tilde{J}^T {(Z^T Z)}^{-1}J & \tilde{J}^T {(Z^T Z)}^{-1} \tilde{J}
        \end{pmatrix}.
    \end{aligned}
\end{equation}
It follows that
\begin{equation}
    \begin{aligned}
        &{\big( J^T {(Z^T Z)}^{-1}J \big)}^{-1}\\
        =&J^T Z^T ZJ - J^T Z^T Z\tilde{J}{(\tilde{J}^T Z^T Z \tilde{J})}^{-1}
            \tilde{J}^T Z^T ZJ \\
        =& J^T Z^T( I_p- Z\tilde{J}{(\tilde{J}^T Z^T Z \tilde{J})}^{-1}
            \tilde{J}^T Z^T) ZJ 
    \end{aligned}
\end{equation}
It follows that
\begin{equation}
    \begin{aligned}
        T(Z)=
        \lambda_{\max}\Big(C^T{\big( J^T {(Z^T Z)}^{-1}J \big)}^{-1}C\Big)\\
    \end{aligned}
\end{equation}
\begin{proposition}\label{optProp}
    Suppose $A$ is a $p\times r$ matrix with rank $r$ and $B$ is a $p\times p$  non-zero semi-definite matrix.
    Let $H_A=A{(A^TA)}^{-1}A^T$.
    Then
    \begin{equation}
        \max_{a^T a=1, a^T A A^T a=0}a^T B a=
        \lambda_{\max}\big((I_p-H_A)B(I_p-H_A)\big).
    \end{equation}
\end{proposition}
\begin{proof}
    Note that $a^T A A^T a=0$ is equivalent to $A^T a=0$ and is in turn equivalent to $H_A a=0$.
    In this circumstance, $a= (I_p-H_A)a$.
    Then
    \begin{equation}\label{eq:prop1eq1}
        \begin{aligned}
        \max_{a^T a=1, a^T A A^T a=0}a^T B a
            &=
            \max_{a^T a=1, H_A a=0}a^T B a\\
            &=
        \max_{a^T a=1, H_A a=0}a^T(I_p-H_A) B (I_p-H_A)a.
        \end{aligned}
    \end{equation}
    It's obvious that $\eqref{eq:prop1eq1}\leq\lambda_{\max}\big((I-H_A)B(I-H_A)\big)$.
    On the other hand, let $\alpha_1$ be one eigenvector corresponding to the largest eigenvalue of $(I-H_A)B(I-H_A)$.
    Note that the row of $H_A$ are all eigenvetors of $(I-H_A)B(I-H_A)$ corresponding to eigenvalue $0$. It follows that $H_A\alpha_1=0$. Now that $\alpha_1$ satisfies the constraint of~\eqref{eq:prop1eq1},~\eqref{eq:prop1eq1} is maximized when $a=\alpha_1$.
    
\end{proof}

\section{Schott's method}

$$
E=ZZ^T-\sum_{i=1}^k n_i \bar{X}_i \bar{X}_i^T.
$$

$$
H=\sum_{i=1}^{k} n_i \bar{X}_i \bar{X}_i^T - n\bar{X}\bar{X}^T.
$$

$$
\mytr E = \mytr Z^T Z - \mytr J^T Z^T Z J.
$$


$$
\mytr H = \mytr J^T Z^T Z J - \frac{1}{n} 1_n^T Z^T Z 1_n
$$

$$
T_{SC}=\frac{1}{\sqrt{n-1}}(
\frac{1}{k-1}\mytr H-\frac{1}{n-k} \mytr E
)
$$



\section*{References}

\bibliography{mybibfile}

\end{document}
