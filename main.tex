\documentclass[review]{elsarticle}

\usepackage{lineno,hyperref}
\modulolinenumbers[5]

\journal{Journal of \LaTeX\ Templates}

%%%%%%%%%%%%%%%%%%%%%%%
%% Elsevier bibliography styles
%%%%%%%%%%%%%%%%%%%%%%%
%% To change the style, put a % in front of the second line of the current style and
%% remove the % from the second line of the style you would like to use.
%%%%%%%%%%%%%%%%%%%%%%%

%% Numbered
%\bibliographystyle{model1-num-names}

%% Numbered without titles
%\bibliographystyle{model1a-num-names}

%% Harvard
%\bibliographystyle{model2-names.bst}\biboptions{authoryear}

%% Vancouver numbered
%\usepackage{numcompress}\bibliographystyle{model3-num-names}

%% Vancouver name/year
%\usepackage{numcompress}\bibliographystyle{model4-names}\biboptions{authoryear}

%% APA style
%\bibliographystyle{model5-names}\biboptions{authoryear}

%% AMA style
%\usepackage{numcompress}\bibliographystyle{model6-num-names}

%% `Elsevier LaTeX' style
\bibliographystyle{elsarticle-num}
%%%%%%%%%%%%%%%%%%%%%%%
\usepackage{xeCJK}
\usepackage{bm}
\usepackage{amsmath}
\usepackage{amssymb}
\usepackage{amsthm}
\usepackage{graphicx}
\usepackage{color}
\usepackage{booktabs}
\usepackage{algorithm}
\usepackage{algorithmic}
\usepackage{multirow}


\DeclareMathOperator{\mytr}{tr}
\DeclareMathOperator{\mydiag}{diag}
\DeclareMathOperator{\myrank}{Rank}
\DeclareMathOperator{\myE}{E}
\DeclareMathOperator{\myVar}{Var}


\theoremstyle{plain}
\newtheorem{theorem}{\quad\quad Theorem}
\newtheorem{proposition}{\quad\quad Proposition}
\newtheorem{corollary}{\quad\quad Corollary}
\newtheorem{lemma}{Lemma}
\newtheorem{example}{Example}
\newtheorem{assumption}{\quad\quad Assumption}
\newtheorem{condition}{Condition}

\theoremstyle{definition}
\newtheorem{remark}{\quad\quad Remark}
\theoremstyle{remark}


\begin{document}

\begin{frontmatter}

\title{A generalized likelihood ratio test for multivariate analysis of variance in high dimension}
\tnotetext[mytitlenote]{Fully documented templates are available in the elsarticle package on \href{http://www.ctan.org/tex-archive/macros/latex/contrib/elsarticle}{CTAN}.}

%% Group authors per affiliation:
\author{Elsevier\fnref{myfootnote}}
\address{Radarweg 29, Amsterdam}
\fntext[myfootnote]{Since 1880.}

%% or include affiliations in footnotes:
\author[mymainaddress,mysecondaryaddress]{Elsevier Inc}
\ead[url]{www.elsevier.com}

\author[mysecondaryaddress]{Global Customer Service\corref{mycorrespondingauthor}}
\cortext[mycorrespondingauthor]{Corresponding author}
\ead{support@elsevier.com}

\address[mymainaddress]{1600 John F Kennedy Boulevard, Philadelphia}
\address[mysecondaryaddress]{360 Park Avenue South, New York}

\begin{abstract}
    This paper considers in the high dimensional setting a canonical testing problem, namely testing the equality of multiple mean vectors of normal distribution.
    By a strategy similar to Roy's union-intersection test, we propose a generalized likelihood ratio test.
    The critical value is determined by permutation method, which produces an exact test.
    The limiting distribution of the test statistic is analysed under non-spiked and spiked covariance.
    Theoretical results and simulation studies show that the test is particularly powerful under spiked covariance.
\end{abstract}

\begin{keyword}
\texttt{elsarticle.cls}\sep \LaTeX\sep Elsevier \sep template
\MSC[2010] 00-01\sep  99-00
\end{keyword}

\end{frontmatter}

\linenumbers
\section{Introduction}
Suppose we have independent observations $X_{ij}\in \mathbb{R}^p$ ($j=1,\ldots,n_k$; $i=1,\ldots, K$) with distribution $N_p(\mu_i,\Sigma)$, where $\mu_i$, $i=1,\ldots,K$, and $\Sigma>0$ are unknown. We would like to test
\begin{equation}\label{hypothesis}
    H: \mu_1=\mu_2=\cdots=\mu_k\quad \textrm{v.s.}\quad K: \textrm{$\mu_i\neq \mu_j$ for some $i\neq j$}.
\end{equation}
The problem is known as one-way multivariate analysis of variance (MANOVA).
A commonly used test for hypothesis~\eqref{hypothesis} is the so-called Wilks' Lambda which is also the LRT. 


In some modern scientific applications, people would like to test hypothesis~\eqref{hypothesis} in high dimensional setting, i.e., $p$ is greater than $n=\sum_{i=1}^{K}n_i$. See, for example,~\cite{Tsai2009}.
However, when $p>n-K$, the LRT for hypothesis~\eqref{hypothesis} is not well defined.
  Researchers have done extensive work to study the testing problem~\eqref{hypothesis} in high dimensional setting.
 Sor far, most tests in the literature are designed for two sample case, i.e. $K=2$.
  See, for example,~\cite{Bai1996Efiect},~\cite{Chen2010A},~\cite{Srivastava2009A},~\cite{Feng2015Multivariate} and~\cite{Tony2013}.
  For the multiple sample case,~\cite{Schott2007Some} modified the Dempster's trace test and proposed a test statistic
  $$
  T_{SC}=\frac{1}{\sqrt{n-1}}\big(
  \frac{1}{K-1}\mytr\big(\sum_{i=1}^K n_i\bar{X}_i\bar{X}_i^T-n\bar{X}\bar{X}^T\big)-\frac{1}{n-K}\mytr\big(\sum_{i=1}^K \sum_{j=1}^{n_i}X_{ij}X_{ij}^T-\sum_{i=1}^K n_i\bar{X}_i\bar{X}_i^T\big)
  \big),
  $$
  where $\bar{X}_i=n_i^{-1}\sum_{j=1}^{n_i}X_{ij}$ and $\bar{X}=n^{-1}\sum_{i=1}^K\sum_{j=1}^{n_i}X_{ij}$. In another work,~\cite{Cai2014High} proposed a test statistic
  $$
  T_{CX}=\max_{1\leq i\leq p} \sum_{1\leq j<l\leq K}\frac{n_j n_l}{n_j+n_l}\frac{(\Omega(\bar{X}_j-\bar{X}_l))_i^2}{\omega_{ii}},
  $$
  Where $\Omega=(\omega)_{ij}=\Sigma^{-1}$ is the precision matrix. When $\Omega$ is unknown, they substitute it by an estimator $\hat{\Omega}$.

  %Suppose $\{X_{i1},\ldots, X_{in_i}\}$ are i.i.d.\ distributed as $N(\mu_i,\Sigma)$ for $1\leq i\leq K$.
Let $\mathbf{X}_i=(X_{i1},\ldots,X_{in_i})$ for $i=1,\ldots,K$.
%The $k$ samples are independent.
%%$\mu_i$, $i=1\ldots, k$ and $\Sigma>0$ are unknown. An interesting problem in multivariate analysis is to test the hypotheses
%\begin{equation}
    %H: \mu_1=\mu_2=\cdots=\mu_k\quad v.s.\quad K: \textrm{$\mu_i\neq \mu_j$ for some $i\neq j$}.
%\end{equation}

%The likelihood ratio test (LRT) has a dominated position in classical multivariate analysis.
   However, most of existing high dimensional tests are not likelihood-based.
    A natural question is how to construct likelihood-based tests in high dimensional setting.
    In a recent work,~\cite{Zhao2016A} proposed a generalized likelihood ratio test in the context of one-sample test for mean vector.
    Their simulation results showed that their test has particular good power performance when the variables are dependent.
    The goal of this paper is to generalize their methodology to MANOVA problem.
    
\section{GLRT}
Let $\mathbf{Z}=(X_1,\ldots,X_k)$.
$$
f(Z;\mu_1,\ldots,\mu_k,\Sigma)=\prod_{i=1}^k\Big[
    (2\pi)^{-n_i p/2}|\Sigma|^{-n_i/2}\exp(-\frac{1}{2}\mathrm{tr}\Sigma^{-1}\sum_{j=1}^{n_i}(x_{ij}-\mu_i)(x_{ij}-\mu_i)^T)
    \Big].
$$
Assume $n=\sum_{i=1}^p n_i<p$. Let $a\in \mathbb{R}^p$ be a vector satisfying $a^T a=1$. Then
$$
f_a(a^T Z;\mu_1,\ldots,\mu_k,\Sigma)=
    (2\pi)^{-n/2}|a^T \Sigma a|^{-n/2}\exp\Big(-\frac{1}{2 a^T \Sigma a}\sum_{i=1}^k\sum_{j=1}^{n_i}(a^Tx_{ij}-a^T\mu_i)^2\Big)
$$
\begin{equation}
    \begin{aligned}
        \max_{\mu_1,\ldots,\mu_k,\Sigma}f_a(a^T Z,\mu_1,\ldots,\mu_k,\Sigma)
        =
        (2\pi)^{-n/2}\big(\sum_{i=1}^k\sum_{j=1}^{n_i}(a^Tx_{ij}-a^T\bar{\mathbf{X}}_i)^2\big)^{-n/2}e^{-{n}/{2}}
    \end{aligned}
\end{equation}

Let $S_i=\sum_{j=1}^{n_i}(x_{ij}-\bar{\mathbf{X}}_i)(x_{ij}-\bar{\mathbf{X}}_i)^T$ and $S=\sum_{i=1}^k S_i$.


Under $H$, we have
\begin{equation}
    \begin{aligned}
        \max_{\mu,\Sigma}f_a(a^T Z,\mu,\ldots,\mu,\Sigma)
        =
        (2\pi)^{-n/2}\big(\sum_{i=1}^k\sum_{j=1}^{n_i}(a^Tx_{ij}-a^T\bar{\mathbf{X}})^2\big)^{-n/2}e^{-{n}/{2}}
    \end{aligned}
\end{equation}

The generalized likelihood ratio test statistic is defined as
\begin{equation}
    \begin{aligned}
        T(Z)
        =
        \max_{a^T a=1, a^T S a=0} 
        a^T \sum_{i=1}^k n_i (\bar{\mathbf{X}}_i-\bar{\mathbf{X}})(\bar{\mathbf{X}}_i-\bar{\mathbf{X}})^T a
    \end{aligned}
\end{equation}
Let $J=\mathrm{diag}(n_1^{-1/2}\mathbf{1}_{n_1},\ldots,n_k^{-1/2}\mathbf{1}_{n_k})$.
Then $S=Z(I_n-JJ^T)Z^T$ and
\begin{equation}
    \begin{aligned}
        \sum_{i=1}^k n_i (\bar{\mathbf{X}}_i-\bar{\mathbf{X}})(\bar{\mathbf{X}}_i-\bar{\mathbf{X}})^T 
        =Z(JJ^T-\frac{1}{n}\mathbf{1}_n\mathbf{1}_n^T)Z^T.
    \end{aligned}
\end{equation}
The matrix $I_n-JJ^T$, $JJ^T-\frac{1}{n}\mathbf{1}_n\mathbf{1}_n^T$ and $\frac{1}{n}\mathbf{1}_n\mathbf{1}_n^T$ are all projection matrix and pairwise orthogonal with rank $n-k$, $k-1$ and $1$.

Let $\tilde{J}$ be a $n\times (n-k)$ matrix satisfied $\tilde{J}\tilde{J}^T =I-JJ^T$.
Then $S=Z\tilde{J}\tilde{J}^T Z^T$ and
 Note that 
$$
Z(JJ^T-\frac{1}{n}\mathbf{1}_n\mathbf{1}_n^T)Z^T
=ZJ(I_k-\frac{1}{n}J^T\mathbf{1}_n \mathbf{1}_n^T J)J^T Z^T.
$$
Note that $I_k-\frac{1}{n}J^T\mathbf{1}_n \mathbf{1}_n^T J$ is a projection matrix with rank $k-1$.
Let $C$ be a $k\times (k-1)$ matrix satisfied $CC^T=I_k-\frac{1}{n}J^T\mathbf{1}_n \mathbf{1}_n^T J$.

In Proposition~\ref{optProp}, letting $A=Z\tilde{J}$ and $B=ZJ C C^T J^TZ^T$ yields 
$$
\begin{aligned}
    T(Z)&=\lambda_{max}\big((I_p-
    Z\tilde{J}{(\tilde{J}^T Z^T Z\tilde{J})}^{-1}\tilde{J}^T Z^T
    )ZJCC^TJ^TZ^T (I_p-
    Z\tilde{J}{(\tilde{J}^T Z^T Z\tilde{J})}^{-1}\tilde{J}^T Z^T
    )\big)
\\
    &=\lambda_{max}\big(C^TJ^TZ^T (I_p-
    Z\tilde{J}{(\tilde{J}^T Z^T Z\tilde{J})}^{-1}\tilde{J}^T Z^T
    )ZJC\big).
\end{aligned}
$$
%where $H_A=Z\tilde{J}{(\tilde{J}^T Z^T Z\tilde{J})}^{-1}\tilde{J}^T Z^T$.
Note that
\begin{equation}
    \begin{aligned}
        &\Big(
        \begin{pmatrix}
            J^T\\
            \tilde{J}^T
        \end{pmatrix}
        Z^T Z
        \begin{pmatrix}
            J&\tilde{J}
        \end{pmatrix}
        \Big)^{-1}\\
        &=
        \begin{pmatrix}
            J^T Z^T ZJ & J^T Z^T Z\tilde{J}\\
            \tilde{J}^T Z^T ZJ & \tilde{J}^T Z^T Z \tilde{J}
        \end{pmatrix}^{-1}
        &=
        \begin{pmatrix}
            J^T {(Z^T Z)}^{-1}J & J^T {(Z^T Z)}^{-1}\tilde{J}\\
            \tilde{J}^T {(Z^T Z)}^{-1}J & \tilde{J}^T {(Z^T Z)}^{-1} \tilde{J}
        \end{pmatrix}.
    \end{aligned}
\end{equation}
It follows that
\begin{equation}
    \begin{aligned}
        &{\big( J^T {(Z^T Z)}^{-1}J \big)}^{-1}\\
        =&J^T Z^T ZJ - J^T Z^T Z\tilde{J}{(\tilde{J}^T Z^T Z \tilde{J})}^{-1}
            \tilde{J}^T Z^T ZJ \\
        =& J^T Z^T( I_p- Z\tilde{J}{(\tilde{J}^T Z^T Z \tilde{J})}^{-1}
            \tilde{J}^T Z^T) ZJ 
    \end{aligned}
\end{equation}
It follows that
\begin{equation}
    \begin{aligned}
        T(Z)=
        \lambda_{\max}\Big(C^T{\big( J^T {(Z^T Z)}^{-1}J \big)}^{-1}C\Big)\\
    \end{aligned}
\end{equation}
\begin{proposition}\label{optProp}
    Suppose $A$ is a $p\times r$ matrix with rank $r$ and $B$ is a $p\times p$  non-zero semi-definite matrix.
    Let $H_A=A{(A^TA)}^{-1}A^T$.
    Then
    \begin{equation}
        \max_{a^T a=1, a^T A A^T a=0}a^T B a=
        \lambda_{\max}\big((I_p-H_A)B(I_p-H_A)\big).
    \end{equation}
\end{proposition}
\begin{proof}
    Note that $a^T A A^T a=0$ is equivalent to $A^T a=0$ and is in turn equivalent to $H_A a=0$.
    In this circumstance, $a= (I_p-H_A)a$.
    Then
    \begin{equation}\label{eq:prop1eq1}
        \begin{aligned}
        \max_{a^T a=1, a^T A A^T a=0}a^T B a
            &=
            \max_{a^T a=1, H_A a=0}a^T B a\\
            &=
        \max_{a^T a=1, H_A a=0}a^T(I_p-H_A) B (I_p-H_A)a.
        \end{aligned}
    \end{equation}
    It's obvious that $\eqref{eq:prop1eq1}\leq\lambda_{\max}\big((I-H_A)B(I-H_A)\big)$.
    On the other hand, let $\alpha_1$ be one eigenvector corresponding to the largest eigenvalue of $(I-H_A)B(I-H_A)$.
    Note that the row of $H_A$ are all eigenvetors of $(I-H_A)B(I-H_A)$ corresponding to eigenvalue $0$. It follows that $H_A\alpha_1=0$. Now that $\alpha_1$ satisfies the constraint of~\eqref{eq:prop1eq1},~\eqref{eq:prop1eq1} is maximized when $a=\alpha_1$.
    
\end{proof}

\section{Schott's method}

$$
E=ZZ^T-\sum_{i=1}^k n_i \bar{X}_i \bar{X}_i^T.
$$

$$
H=\sum_{i=1}^{k} n_i \bar{X}_i \bar{X}_i^T - n\bar{X}\bar{X}^T.
$$

$$
\mytr E = \mytr Z^T Z - \mytr J^T Z^T Z J.
$$


$$
\mytr H = \mytr J^T Z^T Z J - \frac{1}{n} 1_n^T Z^T Z 1_n
$$

$$
T_{SC}=\frac{1}{\sqrt{n-1}}(
\frac{1}{k-1}\mytr H-\frac{1}{n-k} \mytr E
)
$$


\section{Theory}
Let $\Sigma= U\Lambda U^T$ be the eigenvalue decomposition of $\Sigma$, where $\Lambda =\mydiag (\lambda_1,\ldots,\lambda_p)$.
Let $U=(U_1,U_2)$ where $U_1$ is $p\times r$ and $U_2$ is $p\times (p-r)$. Let $\Lambda_1=\mydiag(\lambda_1,\ldots,\lambda_r)$ and $\Lambda_2=\mydiag(\lambda_{r+1},\ldots,\lambda_p)$.
Then $\Sigma=U_1\Lambda_1 U_1^T+U_2\Lambda_2 U_2^T$.

Let $Z\tilde{J}=U_{Z\tilde{J}}D_{Z\tilde{J}}V_{Z\tilde{J}}^T$ be the singular value decomposition of $Z\tilde{J}$. Let $H_{Z\tilde{J}}=U_{Z\tilde{J}}U_{Z\tilde{J}}^T$.
Then
$T(Z) = \lambda_{\max}(C^T J^T Z^T (I_p-H_{Z\tilde{J}})ZJC)$.
Note that
$$
\myE (ZJC) =(\sqrt{n_1}\mu_1,\ldots,\sqrt{n_k}\mu_k) C\overset{def}{=}\mu_{f}.
$$


\begin{assumption}\label{assumpEigen}
    Assume $C \geq \lambda_{r+1} \geq \ldots \geq \lambda_{p} \geq c$, where $c$ and $C$ are absolute constant.
\end{assumption}







\begin{theorem}\label{thm1}
    Suppose Assumption~\eqref{assumpEigen} holds. Suppose 
    \begin{equation}
    p/n\to \infty,\quad\textrm{and}\quad \frac{\lambda_1^2 p}{\lambda_r^2 n^2}\to 0.
    \end{equation}
    Suppose
    \begin{equation}
        \frac{\lambda_r n}{p}\to \infty.
    \end{equation}
    Suppose
    \begin{equation}
        \frac{1}{\sqrt{p}}\|\mu_f\|_F^2=O(1).
    \end{equation}
    Then
    \begin{equation}
        (\mytr \Lambda_2^2)^{-1/2}\big( C^TJ^T Z^T(I_p-H_{Z\tilde J}) ZJC-(\mytr \Lambda_2) I_{k-1} -\mu_f^T(I_p-H_{Z\tilde Z})\mu_f\big)\xrightarrow{\mathcal{L}} W_{k-1},
    \end{equation}
where $W_{k-1}$ is a $(k-1)\times(k-1)$ symmetric random matrix whose entries above the main diagonal are i.i.d.\ $N(0,1)$ and the entries on the diagonal are i.i.d.\ $N(0,2)$.
\end{theorem}

\section{Simulation Results}

In this section, we evaluate the numerical performance of the new test. For comparison, we also carried out simulation for the test of Tony Cai and Yin Xia and the test of Schott. These tests are denoted respectively by NEW, CX and SC.

In the simulations, we set $k=3$.
Note that the new test is invariant under orthogonal transformation.
Without loss of generality, we only consider diagonal $\Sigma$.
We set $\Sigma=\mydiag(p,1,\ldots,1)$.
Define signal-to-noise ratio (SNR) as
$$
\textrm{SNR}=\frac{\|\mu_f\|_F^2}{\sqrt{\sum_{i=2}^{p}\lambda_i(\Sigma)^2}}.
$$
We use SNR to characterize the signal strength.
We consider two alternative hypotheses: the non-sparse alternative and the sparse alternative.
In the non-sparse case, we set $\mu_1=\kappa 1_p$, $\mu_2=-\kappa 1_p$ and $\mu_3=0_p$, where $\kappa$ is selected to make the SNR equal to the given value.
In the sparse case, we set $\mu_1=\kappa (1_{p/5}^T,0_{4p/5}^T)^T$, $\mu_2=\kappa (0_{p/5}^T, 1_{p/5}^T,0_{3p/5}^T)^T$ and $\mu_3=0_p$. Again, $\kappa$ is selected to make the SNR equal to the given value.

%$$
%SNR=\frac{\|\mu_f\|_F^2}{\sqrt{\mytr (\Sigma^2)}}
%$$


\begin{table}[!hbp]
    \caption{Empirical powers of tests under non-sparse alternative with $\alpha=0.05$, $k=3$, $n_1=n_2=n_3=10$. Based on $1000$ replications.}
    \centering
    \begin{tabular}{*{10}{c}}
    \toprule
    \multirow{2}{*}{SNR} &\multicolumn{3}{c}{$p=50$}&\multicolumn{3}{c}{$p=75$}&\multicolumn{3}{c}{$p=100$} \\
        \cmidrule(r){2-4}\cmidrule(r){5-7}\cmidrule(r){8-10}
        &SC & CX & NEW& SC & CX & NEW &SC & CX & NEW\\
    \midrule
0 & 0.035 & 0.048 & 0.052 & 0.057 & 0.052 & 0.057 & 0.053 & 0.048 & 0.045 \\ 
1 & 0.060 & 0.049 & 0.096 & 0.081 & 0.050 & 0.092 & 0.063 & 0.062 & 0.085 \\ 
2 & 0.100 & 0.058 & 0.140 & 0.073 & 0.045 & 0.169 & 0.086 & 0.055 & 0.171 \\ 
3 & 0.145 & 0.066 & 0.234 & 0.119 & 0.070 & 0.266 & 0.117 & 0.056 & 0.307 \\ 
4 & 0.126 & 0.064 & 0.317 & 0.121 & 0.059 & 0.380 & 0.122 & 0.061 & 0.402 \\ 
5 & 0.179 & 0.072 & 0.392 & 0.178 & 0.068 & 0.541 & 0.141 & 0.071 & 0.579 \\ 
6 & 0.198 & 0.070 & 0.513 & 0.189 & 0.071 & 0.639 & 0.143 & 0.066 & 0.717 \\ 
7 & 0.249 & 0.085 & 0.629 & 0.227 & 0.084 & 0.777 & 0.206 & 0.073 & 0.822 \\ 
8 & 0.268 & 0.092 & 0.685 & 0.252 & 0.084 & 0.822 & 0.217 & 0.078 & 0.894 \\ 
9 & 0.324 & 0.100 & 0.786 & 0.256 & 0.090 & 0.911 & 0.246 & 0.074 & 0.949 \\ 
10 & 0.342 & 0.115 & 0.828 & 0.303 & 0.097 & 0.937 & 0.270 & 0.075 & 0.973 \\ 
\bottomrule
\end{tabular}
\end{table}

\begin{table}[!hbp]
    \caption{Empirical powers of tests under non-sparse alternative with $\alpha=0.05$, $k=3$, $n_1=n_2=n_3=25$. Based on $1000$ replications.}
\centering
\begin{tabular}{*{10}{c}}
\toprule
\multirow{2}{*}{SNR} &\multicolumn{3}{c}{$p=100$}&\multicolumn{3}{c}{$p=150$}&\multicolumn{3}{c}{$p=200$} \\
    \cmidrule(r){2-4}\cmidrule(r){5-7}\cmidrule(r){8-10}
    &SC & CX & NEW& SC & CX & NEW &SC & CX & NEW\\
\midrule
0 & 0.050 & 0.043 & 0.050 & 0.056 & 0.066 & 0.048 & 0.062 & 0.045 & 0.054 \\ 
1 & 0.069 & 0.048 & 0.063 & 0.046 & 0.052 & 0.091 & 0.068 & 0.048 & 0.095 \\ 
2 & 0.097 & 0.046 & 0.131 & 0.086 & 0.053 & 0.164 & 0.068 & 0.057 & 0.173 \\ 
3 & 0.113 & 0.061 & 0.200 & 0.117 & 0.057 & 0.270 & 0.101 & 0.045 & 0.313 \\ 
4 & 0.135 & 0.053 & 0.247 & 0.130 & 0.054 & 0.402 & 0.118 & 0.066 & 0.485 \\ 
5 & 0.158 & 0.065 & 0.357 & 0.134 & 0.066 & 0.526 & 0.134 & 0.073 & 0.616 \\ 
6 & 0.198 & 0.081 & 0.433 & 0.161 & 0.052 & 0.668 & 0.138 & 0.067 & 0.765 \\ 
7 & 0.217 & 0.068 & 0.514 & 0.191 & 0.067 & 0.759 & 0.174 & 0.068 & 0.862 \\ 
8 & 0.229 & 0.063 & 0.582 & 0.223 & 0.075 & 0.853 & 0.187 & 0.060 & 0.927 \\ 
9 & 0.264 & 0.094 & 0.680 & 0.218 & 0.080 & 0.918 & 0.227 & 0.067 & 0.966 \\ 
10 & 0.298 & 0.091 & 0.758 & 0.245 & 0.076 & 0.934 & 0.228 & 0.052 & 0.982 \\ 
\bottomrule
\end{tabular}
\end{table}

\begin{table}[!hbp]
\caption[short]{$n_1=n_2=n_3=10$, sparse}
\centering
\begin{tabular}{*{10}{c}}
\toprule
\multirow{2}{*}{SNR} &\multicolumn{3}{c}{$p=50$}&\multicolumn{3}{c}{$p=75$}&\multicolumn{3}{c}{$p=100$} \\
    \cmidrule(r){2-4}\cmidrule(r){5-7}\cmidrule(r){8-10}
    &SC & CX & NEW& SC & CX & NEW &SC & CX & NEW\\
\midrule
    0& 2.000&3.000&4.000&5.000&6.000&7.000&8.000&9.000&10.000\\
0.1\\
0.2\\
\bottomrule
\end{tabular}
\end{table}

\begin{table}[!hbp]
\caption[short]{$n_1=n_2=n_3=25$, sparse}
\begin{center}
\begin{tabular}{*{10}{c}}
\toprule
\multirow{2}{*}{SNR} &\multicolumn{3}{c}{$p=100$}&\multicolumn{3}{c}{$p=150$}&\multicolumn{3}{c}{$p=200$} \\
    \cmidrule(r){2-4}\cmidrule(r){5-7}\cmidrule(r){8-10}
    &SC & CX & NEW& SC & CX & NEW &SC & CX & NEW\\
\midrule
    0& 2.000&3.000&4.000&5.000&6.000&7.000&8.000&9.000&10.000\\
0.1\\
0.2\\
\bottomrule
\end{tabular}
\end{center}
\end{table}

\section{Appendix}

\begin{proof}[\textrm{Proof of Theorem~\ref{thm1}}]
It can be seen that $ZJC$ is independent of ${Z\tilde{J}}$.
Since
$
\myE (Z\tilde{J}) = O_{p\times (n-k)}
$,
we can write
$
Z\tilde{J} = U\Lambda^{1/2} G_1
$,
where $G_1$ is a $p\times (n-k)$ matrix with i.i.d.\ $N(0,1)$ entries.
We write
$
ZJC = \mu_f + U\Lambda^{1/2} G_2
$, 
where $G_2$ is a $p\times (k-1)$ matrix with i.i.d. $N(0,1)$ entries.

Then 
\begin{equation}\label{eq:maindec}
\begin{aligned}
C^TJ^T Z^T(I_p-H_{Z\tilde J}) ZJC
=&
G_2^T \Lambda^{1/2}U^T (I_P-H_{Z\tilde{J}})U\Lambda_{1/2}G_2+
\mu_f^T (I_p -H_{Z\tilde{J}})\mu_f+\\
&\mu_f^T (I_p -H_{Z\tilde{J}})U\Lambda^{1/2}G_2+
G_2^T \Lambda^{1/2}U^T (I_P-H_{Z\tilde{J}})\mu_f.
\end{aligned}
\end{equation}
To deal the first term, we note that
$$
G_2^T \Lambda^{1/2}U^T (I_p-H_{Z\tilde{J}})U\Lambda_{1/2}G_2\sim
\sum_{i=1}^p \lambda_i (\Lambda^{1/2}U^T (I_p-H_{Z\tilde{J}})U\Lambda^{1/2})\xi_i \xi_i^T,
$$
where $\xi_i\overset{i.i.d.}{\sim} N(0,I_{k-1})$. The key to its asymptotic behavior is the positive eigenvalues of $\Lambda^{1/2}U^T (I_p-H_{Z\tilde{J}})U\Lambda^{1/2}$, which in turn equal to the eigenvalues of $(I_p-H_{Z\tilde{J}})U\Lambda U^T (I_p-H_{Z\tilde{J}})$.
Write $(I_p-H_{Z\tilde{J}})U\Lambda U^T (I_p-H_{Z\tilde{J}})$ as the sum of two terms
$$
\begin{aligned}
&(I_p-H_{Z\tilde{J}})U\Lambda U^T (I_p-H_{Z\tilde{J}})
\\
=&
(I_p-H_{Z\tilde{J}})U_1\Lambda_1 U_1^T(I_p-H_{Z\tilde{J}})+(I_p-H_{Z\tilde{J}})U_2\Lambda_2 U_2^T (I_p-H_{Z\tilde{J}})
\overset{def}{=}R_1+R_2.
\end{aligned}
$$

Note that
$$
\begin{aligned}
&\lambda_{\max}\big( R_1 \big)
=
\lambda_{\max}\big(\Lambda_1^{1/2} U_1^T(I_p-H_{Z\tilde{J}}) U_1 \Lambda_1^{1/2}\big)
\leq 
\lambda_{\max}\big(\Lambda_1^{1/2} U_1^T(I_p-U_{Z\tilde{J}[,1:r]}U_{Z\tilde{J}[,1:r]}^T) U_1 \Lambda_1^{1/2}\big)\\
\leq &
\lambda_1
\lambda_{\max}\big(U_1^T(I_p-U_{Z\tilde{J}[,1:r]}U_{Z\tilde{J}[,1:r]}^T) U_1 \big)
= 
\lambda_1
\lambda_{\max}\big(I_r - U_1^TU_{Z\tilde{J}[,1:r]}U_{Z\tilde{J}[,1:r]}^T U_1 \big).
\end{aligned}
$$

To investigate the behavior of $U_{Z\tilde{J}}$, we need to investigate the behavior of $D_{Z\tilde{J}}$ first.
Note that 
$
G_1^T \Lambda G_1 = \tilde{J}^T Z^T Z\tilde{J} = V_{Z\tilde{J}} D_{Z\tilde{J}}^2 V_{Z\tilde{J}}^T
$, and 
$
G_1^T \Lambda G_1=
G_{1[1:r,]}^T \Lambda_1 G_{1[1:r,]}+
G_{1[(r+1):p,]}^T \Lambda_2 G_{1[(r+1):p,]}
$. We have
$$
V_{Z\tilde{J}} D_{Z\tilde{J}}^2 V_{Z\tilde{J}}^T=
G_{1[1:r,]}^T \Lambda_1 G_{1[1:r,]}+
G_{1[(r+1):p,]}^T \Lambda_2 G_{1[(r+1):p,]}.
$$
For $i=1,\ldots, r$,
\begin{equation}\label{eq:DLower}
\begin{aligned}
&\lambda_i(G_{1[1:r,]}^T \Lambda_1 G_{1[1:r,]})
\geq
\lambda_i(G_{1[1:r,]}^T \mydiag(\lambda_i I_{i},O_{(r-i)\times(r-i)}) G_{1[1:r,]})
\\
= &
\lambda_i \lambda_i(G_{1[1:i,]}G_{1[1:i,]}^T)=\lambda_i n(1+o_P(1)),
\end{aligned}
\end{equation}
where the last equality holds since $n^{-1}G_{1[1:i,]}G_{1[1:i,]}^T\xrightarrow{P}I_i$ by law of large numbers.
On the other hand, for $i=1,\ldots, r$,
\begin{equation}\label{eq:DUpper}
\begin{aligned}
&\lambda_i(G_{1[1:r,]}^T \Lambda_1 G_{1[1:r,]})
\\
=&\lambda_i\Big(
G_{1[1:r,]}^T \big(
\mydiag(\lambda_1,\ldots,\lambda_{i-1},O_{(r-i+1)\times(r-i+1)})+
\mydiag(O_{(i-1)\times(i-1)},\lambda_i,\ldots,\lambda_r)
\big)
G_{1[1:r,]}
\Big)\\
\leq&
\lambda_1(G_{1[1:r,]}^T \mydiag(O_{(i-1)\times(i-1)},\lambda_i,\ldots,\lambda_r) G_{1[1:r,]})
\leq
\lambda_1(G_{1[1:r,]}^T \mydiag(O_{(i-1)\times(i-1)},\lambda_i I_{r-i+1}) G_{1[1:r,]})
\\
= &
\lambda_i \lambda_1(G_{1[i:r,]}G_{1[i:r,]}^T)=\lambda_i n(1+o_P(1))
\end{aligned}
\end{equation}
where the first inequality holds by Weyl's inequality. It follows from~\eqref{eq:DLower} and~\eqref{eq:DUpper} that 
$\lambda_i(G_{1[1:r,]}^T \Lambda_1 G_{1[1:r,]})=\lambda_i n(1+o_P(1))$ for $i=1,\ldots, r$.

Note that
$\lambda_{\max}(G_{1[(r+1):p,]}^T \Lambda_2 G_{1[(r+1):p,]})\leq C\lambda_{\max}(G_{1[(r+1):p,]}^T G_{1[(r+1):p,]})=O_P(p)$ by Bai-Yin's law.
By assumption $\lambda_r n/p\to \infty$, we can deduce that $D_{Z\tilde{J}[i,i]}^2=\lambda_i(G_1^T \Lambda G_1)=\lambda_i n(1+o_P(1))$, $i=1,\ldots, r$.

Now we are ready to investigate the behavior of $U_{Z\tilde{J}}$.
Since
$
U\Lambda^{1/2} G_1 G_1^T \Lambda^{1/2} U^T 
=U_{Z\tilde{J}}D_{Z\tilde{J}}^2 U_{Z\tilde{J}}^T
$,
we have
$
G_1 G_1^T  
=\Lambda^{-1/2} U^T U_{Z\tilde{J}}D_{Z\tilde{J}}^2 U_{Z\tilde{J}}^TU\Lambda^{-1/2}
$, which further indicates
$$
\begin{aligned}
&G_{1[(r+1):p,]} G_{1[(r+1):p,]}^T  
=\Lambda_{2}^{-1/2} U_{[,(r+1):p]}^T U_{Z\tilde{J}}D_{Z\tilde{J}}^2 U_{Z\tilde{J}}^T U_{[,(r+1):p]}\Lambda_{2}^{-1/2}\\
\geq&
\Lambda_{2}^{-1/2} U_{[,(r+1):p]}^T U_{Z\tilde{J}[,1:r]}D_{Z\tilde{J}[1:r,1:r]}^2 U_{Z\tilde{J}[,1:r]}^T U_{[,(r+1):p]}\Lambda_{2}^{-1/2}\\
\geq&
D_{Z\tilde{J}[r,r]}^2
\Lambda_{2}^{-1/2} U_{[,(r+1):p]}^T U_{Z\tilde{J}[,1:r]} U_{Z\tilde{J}[,1:r]}^T U_{[,(r+1):p]}\Lambda_{2}^{-1/2}.
\end{aligned}
$$
Thus,
$$
\lambda_{\max}(U_{[,(r+1):p]}^T U_{Z\tilde{J}[,1:r]} U_{Z\tilde{J}[,1:r]}^T U_{[,(r+1):p]})\leq 
\frac{C}{D^2_{Z\tilde{J}[r,r]}} \lambda_{1}
(G_{1[(r+1):p,]} G_{1[(r+1):p,]}^T)
=O_P(\frac{p}{\lambda_r n}).
$$

Note that we have the simple relationship
$$
\begin{aligned}
&\lambda_{\max}(U_{[,(r+1):p]}^T U_{Z\tilde{J}[,1:r]} U_{Z\tilde{J}[,1:r]}^T U_{[,(r+1):p]})
=
\lambda_{\max}( U_{Z\tilde{J}[,1:r]}^T U_{[,(r+1):p]}U_{[,(r+1):p]}^T U_{Z\tilde{J}[,1:r]})\\
=&
\lambda_{\max}( U_{Z\tilde{J}[,1:r]}^T (I_p- U_1 U_1^T) U_{Z\tilde{J}[,1:r]})=
\lambda_{\max}(I_r- U_{Z\tilde{J}[,1:r]}^T  U_1 U_1^T U_{Z\tilde{J}[,1:r]})\\
=&
1-\lambda_{\min}( U_{Z\tilde{J}[,1:r]}^T  U_1 U_1^T U_{Z\tilde{J}[,1:r]})
=
1-\lambda_{\min}(U_1^T U_{Z\tilde{J}[,1:r]}U_{Z\tilde{J}[,1:r]}^T U_1)\\
=&
\lambda_{\max}(I_r-U_1^T U_{Z\tilde{J}[,1:r]}U_{Z\tilde{J}[,1:r]}^T U_1).
\end{aligned}
$$
Therefore 
$
\lambda_{\max}(I_r-U_1^T U_{Z\tilde{J}[,1:r]}U_{Z\tilde{J}[,1:r]}^T U_1)
=O_P(\frac{ p}{\lambda_r n})
$, and we can conclude
$\lambda_{\max}(R_1)=O_P(\frac{\lambda_1 p}{\lambda_r n})$.

We now deal with $R_1+R_2$.
For $i=1,\ldots, r$,
$$
\lambda_i(R_1+R_2)\leq
\lambda_1(R_1+R_2)\leq \lambda_1(R_1)+\lambda_1(R_2)\leq O_P(\frac{\lambda_1 p}{\lambda_r n}) + C.
$$
For $i=r+1,\ldots, p-r$,
$$
\begin{aligned}
&\lambda_i(R_1+R_2)\leq \lambda_{i-r}(R_2)
=
 \lambda_{i-r}\big( \Lambda_2^{1/2} U_2^T (I_p-H_{Z\tilde{J}})U_2\Lambda_2^{1/2}\big)
\leq
\lambda_{i-r}(\Lambda_2)
=\lambda_i.
\end{aligned}
$$
On the other hand,
for $i=1,\ldots, p-r-n+k$,
$$
\begin{aligned}
&\lambda_i(R_1+R_2)\geq \lambda_i(R_2)
=
 \lambda_i\big( \Lambda_2^{1/2} U_2^T (I_p-H_{Z\tilde{J}})U_2\Lambda_2^{1/2}\big)\\
=&
\lambda_i\big( \Lambda_2- \Lambda_2^{1/2} U_2^T H_{Z\tilde{J}}U_2 \Lambda_2^{1/2}\big)
\geq 
\lambda_{i+n-k}.
\end{aligned}
$$
The last equality holds since $U_2^T H_{Z\tilde{J}}U_2$ is at most of rank $n-k$.

As a consequence of these bounds, we have
$$
\sum_{i=1}^{p-r-n+k}\lambda_{i+n-k}^2\leq \mytr [(R_1+R_2)^2]\leq  r(O_P(\frac{\lambda_1 p}{\lambda_r n})+C)^2+\sum_{i=r+1}^{p-r}\lambda_i^2,
$$
or
$$
| \mytr [(R_1+R_2)^2]-\sum_{i=r+1}^{p}\lambda_{i}^2|\leq 
\sum_{i=r+1}^{n-k}\lambda_{i}^2+
\sum_{i=p-r+1}^{p}\lambda_{i}^2
+
r(O_P(\frac{\lambda_1 p}{\lambda_r n})+C)^2.
$$
Similarly,
$$
| \mytr [(R_1+R_2)]-\sum_{i=r+1}^{p}\lambda_{i}|\leq 
\sum_{i=r+1}^{n-k}\lambda_{i}+
\sum_{i=p-r+1}^{p}\lambda_{i}
+
r(O_P(\frac{\lambda_1 p}{\lambda_r n})+C).
$$
These, conbined with the assumptions, yield
$$
 \mytr [(R_1+R_2)^2]=(1+o_P(1))\sum_{i=r+1}^{p}\lambda_{i}^2,
$$
and
$$
 \mytr [(R_1+R_2)]=\sum_{i=r+1}^{p}\lambda_{i}+O(n)+O_P(\frac{\lambda_1 p}{\lambda_r n}).
$$

Now we have the Lyapunov condition
$$
\frac{\lambda_1[(R_1+R_2)^2]}{
\mytr [(R_1+R_2)^2]
}
%\leq
%\frac{
%\big( O_P(\frac{\lambda_1 p}{\lambda_r n})+C\big)^2
%}{\sum_{i=1}^{p-r-n+k}\lambda_{i+n-k}^2}
%\leq
%\frac{
%\big( O_P(\frac{\lambda_1 p}{\lambda_r n})+C\big)^2
%}{c(p-r-n+k)}
=
\frac{
\big( O_P(\frac{\lambda_1 p}{\lambda_r n})+C\big)^2
}{
(1+o_P(1))\sum_{i=r+1}^{p}\lambda_{i}^2
}
\xrightarrow{P} 0.
$$
Apply Lyapunov central limit theorem conditioning on $H_{Z\tilde{J}}$, we have
$$
\begin{aligned}
\big(\mytr[(R_1+R_2)^2]\big)^{-1/2}
\big( G_2^T \Lambda^{1/2}U^T (I_p-H_{Z\tilde{J}})U\Lambda_{1/2}G_2-\mytr(R_1+R_2) I_{k-1} \big)
\xrightarrow{\mathcal{L}} W_{k-1}
\end{aligned}
$$
where $W_{k-1}$ is a $(k-1)\times(k-1)$ symmetric random matrix whose entries above the main diagonal are i.i.d.\ $N(0,1)$ and the entries on the diagonal are i.i.d.\ $N(0,2)$.
By Slutsky's theorem, we have
$$
\begin{aligned}
\big(\sum_{i=r+1}^p \lambda_i^2\big)^{-1/2}
\big( G_2^T \Lambda^{1/2}U^T (I_p-H_{Z\tilde{J}})U\Lambda_{1/2}G_2-(\sum_{i=r+1}^p \lambda_i)I_{k-1} \big)
\xrightarrow{\mathcal{L}}W_{k-1}
\end{aligned}
$$


As for the cross term of~\eqref{eq:maindec}, we have
$$
\begin{aligned}
    &\myE [\|\mu_f^T (I_p -H_{Z\tilde{J}})U\Lambda^{1/2}G_2\|_F^2|Z\tilde{J}]\\
    = &
    (k-1)\mytr(\mu_f^T (I_p -H_{Z\tilde{J}})U\Lambda U^T (I_p -H_{Z\tilde{J}})\mu_f)\\
    \leq &
    (k-1)\lambda_1\big((I_p -H_{Z\tilde{J}})U\Lambda U^T (I_p -H_{Z\tilde{J}})\big)\|\mu_f\|^2_F\\
    = &
    (k-1) O_P(\frac{\lambda_1 p}{\lambda_r n})  \|\mu_f\|^2_F\\
    = &
    (k-1) O_P(\frac{\lambda_1 \sqrt{p}}{\lambda_r n}) \sqrt{p}  \|\mu_f\|^2_F=o_P(p)
\end{aligned}
$$
The last equality holds when we assume $\frac{1}{\sqrt{p}}\|\mu_f\|_F^2=O(1)$. Hence $\|\mu_f^T (I_p -H_{Z\tilde{J}})U\Lambda^{1/2}G_2\|_F^2=o_P(p)$.
This completes the proof of the theorem.

\end{proof}






\section*{References}

\bibliography{mybibfile}

\end{document}
